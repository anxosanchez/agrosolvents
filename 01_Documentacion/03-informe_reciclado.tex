% Options for packages loaded elsewhere
% Options for packages loaded elsewhere
\PassOptionsToPackage{unicode}{hyperref}
\PassOptionsToPackage{hyphens}{url}
\PassOptionsToPackage{dvipsnames,svgnames,x11names}{xcolor}
%
\documentclass[
  a4paper,
]{report}
\usepackage{xcolor}
\usepackage{amsmath,amssymb}
\setcounter{secnumdepth}{5}
\usepackage{iftex}
\ifPDFTeX
  \usepackage[T1]{fontenc}
  \usepackage[utf8]{inputenc}
  \usepackage{textcomp} % provide euro and other symbols
\else % if luatex or xetex
  \usepackage{unicode-math} % this also loads fontspec
  \defaultfontfeatures{Scale=MatchLowercase}
  \defaultfontfeatures[\rmfamily]{Ligatures=TeX,Scale=1}
\fi
\usepackage{lmodern}
\ifPDFTeX\else
  % xetex/luatex font selection
\fi
% Use upquote if available, for straight quotes in verbatim environments
\IfFileExists{upquote.sty}{\usepackage{upquote}}{}
\IfFileExists{microtype.sty}{% use microtype if available
  \usepackage[]{microtype}
  \UseMicrotypeSet[protrusion]{basicmath} % disable protrusion for tt fonts
}{}
\makeatletter
\@ifundefined{KOMAClassName}{% if non-KOMA class
  \IfFileExists{parskip.sty}{%
    \usepackage{parskip}
  }{% else
    \setlength{\parindent}{0pt}
    \setlength{\parskip}{6pt plus 2pt minus 1pt}}
}{% if KOMA class
  \KOMAoptions{parskip=half}}
\makeatother
% Make \paragraph and \subparagraph free-standing
\makeatletter
\ifx\paragraph\undefined\else
  \let\oldparagraph\paragraph
  \renewcommand{\paragraph}{
    \@ifstar
      \xxxParagraphStar
      \xxxParagraphNoStar
  }
  \newcommand{\xxxParagraphStar}[1]{\oldparagraph*{#1}\mbox{}}
  \newcommand{\xxxParagraphNoStar}[1]{\oldparagraph{#1}\mbox{}}
\fi
\ifx\subparagraph\undefined\else
  \let\oldsubparagraph\subparagraph
  \renewcommand{\subparagraph}{
    \@ifstar
      \xxxSubParagraphStar
      \xxxSubParagraphNoStar
  }
  \newcommand{\xxxSubParagraphStar}[1]{\oldsubparagraph*{#1}\mbox{}}
  \newcommand{\xxxSubParagraphNoStar}[1]{\oldsubparagraph{#1}\mbox{}}
\fi
\makeatother


\usepackage{longtable,booktabs,array}
\usepackage{calc} % for calculating minipage widths
% Correct order of tables after \paragraph or \subparagraph
\usepackage{etoolbox}
\makeatletter
\patchcmd\longtable{\par}{\if@noskipsec\mbox{}\fi\par}{}{}
\makeatother
% Allow footnotes in longtable head/foot
\IfFileExists{footnotehyper.sty}{\usepackage{footnotehyper}}{\usepackage{footnote}}
\makesavenoteenv{longtable}
\usepackage{graphicx}
\makeatletter
\newsavebox\pandoc@box
\newcommand*\pandocbounded[1]{% scales image to fit in text height/width
  \sbox\pandoc@box{#1}%
  \Gscale@div\@tempa{\textheight}{\dimexpr\ht\pandoc@box+\dp\pandoc@box\relax}%
  \Gscale@div\@tempb{\linewidth}{\wd\pandoc@box}%
  \ifdim\@tempb\p@<\@tempa\p@\let\@tempa\@tempb\fi% select the smaller of both
  \ifdim\@tempa\p@<\p@\scalebox{\@tempa}{\usebox\pandoc@box}%
  \else\usebox{\pandoc@box}%
  \fi%
}
% Set default figure placement to htbp
\def\fps@figure{htbp}
\makeatother





\setlength{\emergencystretch}{3em} % prevent overfull lines

\providecommand{\tightlist}{%
  \setlength{\itemsep}{0pt}\setlength{\parskip}{0pt}}



 


\makeatletter
\@ifpackageloaded{caption}{}{\usepackage{caption}}
\AtBeginDocument{%
\ifdefined\contentsname
  \renewcommand*\contentsname{Table of contents}
\else
  \newcommand\contentsname{Table of contents}
\fi
\ifdefined\listfigurename
  \renewcommand*\listfigurename{List of Figures}
\else
  \newcommand\listfigurename{List of Figures}
\fi
\ifdefined\listtablename
  \renewcommand*\listtablename{List of Tables}
\else
  \newcommand\listtablename{List of Tables}
\fi
\ifdefined\figurename
  \renewcommand*\figurename{Figure}
\else
  \newcommand\figurename{Figure}
\fi
\ifdefined\tablename
  \renewcommand*\tablename{Table}
\else
  \newcommand\tablename{Table}
\fi
}
\@ifpackageloaded{float}{}{\usepackage{float}}
\floatstyle{ruled}
\@ifundefined{c@chapter}{\newfloat{codelisting}{h}{lop}}{\newfloat{codelisting}{h}{lop}[chapter]}
\floatname{codelisting}{Listing}
\newcommand*\listoflistings{\listof{codelisting}{List of Listings}}
\makeatother
\makeatletter
\makeatother
\makeatletter
\@ifpackageloaded{caption}{}{\usepackage{caption}}
\@ifpackageloaded{subcaption}{}{\usepackage{subcaption}}
\makeatother
\usepackage{bookmark}
\IfFileExists{xurl.sty}{\usepackage{xurl}}{} % add URL line breaks if available
\urlstyle{same}
\hypersetup{
  colorlinks=true,
  linkcolor={blue},
  filecolor={Maroon},
  citecolor={Blue},
  urlcolor={Blue},
  pdfcreator={LaTeX via pandoc}}


\author{}
\date{}
\begin{document}

\renewcommand*\contentsname{Table of contents}
{
\hypersetup{linkcolor=}
\setcounter{tocdepth}{2}
\tableofcontents
}

\chapter{Simulación de uns planta de reciclado de fames y
microemulsiones}\label{simulaciuxf3n-de-uns-planta-de-reciclado-de-fames-y-microemulsiones}

La simulación de una planta de reciclaje de biodiésel (FAME - Fatty Acid
Myl Esters) que ha sido utilizado como disolvente industrial es un
proyecto de ingeniería química complejo. No obstante, en este proyecto
se trata de analizar e intentar aprovechas las capacidades de la emprese
hermana d DORVI, REGADI, posee en las instalaciones de DROVI y
destinadas al reciclaje de disolventes convencioonales por combinaci´on
de evaporación en un evpapofador de capa fina agitada y una columna de
destilación.

Los FAME poseen un alto punto de ebullición y propiedades termodinámicas
específicas que hacen que su recuperación requiera cuidados para evitar
la degradación térmica. Aquí hay una hoja de ruta estructurada paso a
paso para comenzar su proyecto:

\section{Definición del problema y el caudal de
entrada}\label{definiciuxf3n-del-problema-y-el-caudal-de-entrada}

Antes de abrir cualquier software, necesita datos concretos. El
simulador devolverá basura si introducimos basura (basura que entra,
basura que sale). - Composición del biodiesel: Suele ser una mezcla de
ésteres metílicos (C16, C18, etc.). Para la simulación, el oleato de
metilo se utiliza a menudo como ``pseudocomponente'' representativo si
no se dispone de la cromatografía exacta. - Contaminantes: ¿Qué disolvió
el biodiesel? - ¿Pinturas o resinas? (Sólidos o polímeros) - ¿Aceites
minerales? (Hidrocarburos Pesados) - Agua? - Especificaciones del
Producto: - ¿Qué pureza necesita para reutilizar como solvente? (ej. 99
\% de pureza, \textless0,05 \% de agua).

\section{Selección de operaciones
unitarias}\label{selecciuxf3n-de-operaciones-unitarias}

Dependiendo de los contaminantes, el proceso variará, pero el esquema
clásico para recuperar solventes de alto punto de ebullición suele ser:
- Filtración: si el biodiesel tiene partículas sólidas o lodos
arrastrados. - Decantación (flash): si hay agua o solventes altamente
volátiles mezclados, un tanque flash puede separarlos fácilmente. -
Destilación al vacío: este es el punto crítico. El biodiesel se degrada
(oxida o polimeriza) a altas temperaturas (generalmente por encima de
\(250^\circ\text{C}\)). - Dado que su punto de ebullición a presión
atmosférica es muy alto (\(340-375^\circ\text{C}\)), es obligatorio
utilizar vacío para bajar la temperatura de ebullición y destilarlo sin
``quemarlo''

\section{Elección de software de
simulación}\label{elecciuxf3n-de-software-de-simulaciuxf3n}

Hay varias opciones según el presupuesto y acceso: - Aspen Plus / Aspen
HYSYS: El estándar de la industria. Tiene bases de datos muy completas
para FAMEs. - ChemCAD: Muy potente para procesos químicos y
destilaciones. - DWSIM (Recomendado por precio): Es Open Source y
gratuito. Tiene una buena base de datos y permite simular destilaciones
y reactores. Para un proyecto académico o de creación de prototipos, es
la mejor opción. - ProSim: Otra opción comercial con buenas capacidades
de simulación. - Simuladores específicos de destilación: Algunos
simuladores están especializados en destilación y pueden ser útiles si
la destilación es el foco principal.

En nuestro caso (UVIGO) poseemos software de simulación ASPEN - HYSYS
que es el utilizado en muchos de nuestras simulaciones sobre todo en lo
relacionado acon la viabilidad de la recuperación y la simulación del
reciclado de los palmitatos y las microsemulsiones
pero,desafortunadamente la empresa DROVI no posee licencia de este
software y, dado que la licencia de la UVIGO es de educación, no podemos
pasar los archivos de simulación a entidades privadas, por lo que se
recomienda utilizar DWSIM que es gratuito y de código abierto en sus
simulaciones. No obstante, en este apartado dejaremos las instrucciones
necesarias para la simulación con ASPEN - HYSYS y DWSIM.

\section{Configuración de la
simulación}\label{configuraciuxf3n-de-la-simulaciuxf3n}

\subsection{Modelo Termodinámico (El ``motor'' de la simulación) Esta es
una decisión técnica muy importante. El biodiesel es una molécula polar
y orgánica or lo
que:}\label{modelo-termodinuxe1mico-el-motor-de-la-simulaciuxf3n-esta-es-una-decisiuxf3n-tuxe9cnica-muy-importante.-el-biodiesel-es-una-moluxe9cula-polar-y-orguxe1nica-or-lo-que}

\begin{itemize}
\tightlist
\item
  No se debe usar el modelo de gas ideal.
\item
  Utilizar modelos de coeficientes de actividad como NRTL o UNIQUAC.
  Funcionan muy bien para mezclas líquido-líquido y para predecir el
  equilibrio cuando hay agua o alcoholes presentes.
\item
  Si la mezcla consta solo de hidrocarburos (aceites), Peng-Robinson
  podría funcionar, pero NRTL es más seguro si no está seguro.
\end{itemize}

\section{Pasos para configurar la simulación (Ejemplo en
DWSIM/Aspen)}\label{pasos-para-configurar-la-simulaciuxf3n-ejemplo-en-dwsimaspen}

Una vez elegido el software, siga este orden: - Configuración de
componentes: - Agregar oleato de metilo (como componente base). -
Agregar agua (si hay humedad). - Agregar el contaminante principal (por
ejemplo, tolueno o un componente genérico pesado). - Selección do
Paquete de Fluídos: Selecciona NRTL ou UNIQUAC.

\subsection{Diseño del Diagrama de Flujo
(PFD):}\label{diseuxf1o-del-diagrama-de-flujo-pfd}

\begin{itemize}
\tightlist
\item
  Creara unha Corrente de Alimentación (Feed) coa temperatura e presión
  atmosférica.
\item
  Colocar una Columna de Destilación Rigurosa.
\item
  Configuración de la columna:

  \begin{itemize}
  \tightlist
  \item
    Presión del condensador: Establecer una presión de vacío (por
    ejemplo, \(0,1 \text{ bar}\) o \(10 \text{ kPa}\)).
  \item
    Relación de reflujo: comenzar con algo conservador, como 1,5 o 2.
  \item
    Especificaciones: definir lo que desea obtener del fondo (el
    biodiesel limpio generalmente sale por el fondo si los contaminantes
    son volátiles, o por la parte superior si los contaminantes son
    lodos pesados).
  \end{itemize}
\end{itemize}

\subsection{Resumen de los desafíos
clave}\label{resumen-de-los-desafuxedos-clave}

\begin{itemize}
\tightlist
\item
  \textbf{Nota importante}: a diferencia de la recuperación de acetona o
  etanol, el biodiesel no se evapora fácilmente. Si sus residuos son más
  pesados \hspace{0pt}\hspace{0pt}que el biodiesel (por ejemplo,
  resinas), tendrá que evaporar el biodiesel (mucha energía). Si los
  residuos son más ligeros (por ejemplo, agua, disolventes ligeros),
  basta con decapar o evaporar los ligeros y dejar el biodiesel líquido
  (mucho más eficiente).
\end{itemize}

\chapter{Evaporador de Película Agitada (ATFE - Agitated Thin Film
Evaporator)}\label{evaporador-de-peluxedcula-agitada-atfe---agitated-thin-film-evaporator}

La emppresa REGADI utiliza un evaporador de película agitada (ATFE) para
la recuperación de disolventes. Este tipo de evaporador es especialmente
adecuado para líquidos viscosos y sensibles al calor, como el biodiésel.
El ATFE funciona creando una película delgada de líquido sobre una
superficie calentada, mientras una cuchilla agita el líquido para
mejorar la transferencia de calor y masa. Esto permite una rápida
evaporación del disolvente volátil a temperaturas más bajas, minimizando
el riesgo de degradación térmica del biodiésel. El diseño del ATFE
incluye un rotor que gira a alta velocidad, creando una fuerza
centrífuga que distribuye el líquido en una capa delgada. La agitación
mecánica mejora la transferencia de calor al reducir la resistencia
térmica en la película líquida y facilita la liberación del vapor
formado. De hecho, es el equipo ideal para biodiesel sucio porque
minimiza el tiempo de residencia y evita la degradación térmica
(``quemar'' el biodiesel), además de manejar bien la viscosidad.

No existe una biblioteca Python específica llamada ``modelo ATFE'' o
similar. A diferencia de las redes neuronales o la simulación de datos
(donde tiene scikit-learn), en la ingeniería de procesos en Python
generalmente se debe construir el modelo de operación unitaria por uno
mismo o usar libros genéricos de termodinámica para respaldar sus
cálculos.

Para simular esto en Python, necesitará crear un modelo discreto (paso a
paso) o resolver un sistema de ecuaciones diferenciales ordinarias (EDO)
a lo largo del evaporador. Nosotros lo hemos abordado de la siguiente
forma:

\section{La Física del ATFE}\label{la-fuxedsica-del-atfe}

El ATFE no es un tanque mixto; es un tubo largo. Se debes modelizar
dividiendo el tubo en pequeñas ``rodajas'' o nodos (volumen de control).
En cada rebanada (\(dz\)), suceden tres cosas principales: 1.
Transferencia de calor: desde la pared caliente (\(T_{wall}\)) a la
película líquida (\(T_{liq}\)). 2. Evaporación: Parte del disolvente
volátil se convierte en vapor debido al calor recibido. 3. Cambio de
Composición: El líquido que pasa a la siguiente rebanada es más rico en
biodiesel y más viscoso.

Las variables clave son: - \(U\) (Coeficiente general de transferencia
de calor): en el ATFE, este valor es alto debido a la agitación
mecánica. - \(\lambda\) (Calor latente de vaporización): energía
requerida para evaporar el solvente.

\section{Bibliotecas Python
necesarias}\label{bibliotecas-python-necesarias}

Para hacer esto, necesitaremos: - \textbf{NumPy}: para manejar vectores
y matrices. - \textbf{SciPy} (integrar): para resolver ecuaciones
diferenciales (odeint o solve\_ivp) si quiere ser riguroso, o
simplemente un bucle for si hace un modelo de diferencias finitas. -
\textbf{Thermo} (opcional pero recomendado): una biblioteca Python de
código abierto para obtener propiedades químicas (\(Cp\), puntos de
ebullición, densidad) sin tener que escribirlas a mano. (pipa
instalación termo).

\chapter{Códigos de simulaciónen
python}\label{cuxf3digos-de-simulaciuxf3nen-python}

A partir de esta premisa hemos creado códigos específicos para simular
el comportamiento de cada componente que forma parte de este proyecto de
reciclado de ésteres y microemulsiones en las instalaciones de DROVI y
REGADI. Los códigos están disponibles en el repositorio de GitHub
asociado a este proyecto y se entrega copia digital con esta memoria.

La siumulación \texttt{atfe\_1.py} simula el comportamiento del
evaporador de película agitada (ATFE) para la recuperación de biodiésel
a partir de una mezcla contaminada. Este código utiliza un modelo de
diferencias finitas para dividir el evaporador en segmentos y calcular
la transferencia de calor, la evaporación y el cambio de composición a
lo largo del equipo.

A este modelo le falta mejorar dos puntos en el código, aunque en la
actualidad no se emplean en el reciclaje de disolventes actual por parte
de REGADI, pero que muy probablemente se pueda usar en el futuro es la
destilación de los FAMES recuperados para su purificación final. Ello
imlica tener en cuanta también ell equilibbrio vappor-líquido y la
variación del coeficiente de transferencia de calor \(U\) a lo largo del
evaporador. Estos dos puntos son:

\begin{itemize}
\tightlist
\item
  VLE (Equilibrio vapor-líquido): hemos supuesto que la temperatura de
  ebullición es constante (\(45^\circ\text{C}\)). De hecho, a medida que
  el disolvente se evapora y aumenta la concentración de biodiesel,
  aumenta la temperatura de ebullición de la mezcla (Elevación del punto
  de ebullición). Debe llamar a termo en cada paso para recalcular la
  temperatura de equilibrio (punto de burbuja).
\item
  Viscosidad y U variable: a medida que el biodiesel se concentra, la
  viscosidad se dispara. Esto hace que el coeficiente \(U\) disminuya al
  final de la tubería. Un buen modelo ATFE tiene una fórmula que
  recalcula \(U\) en función de la viscosidad local.
\end{itemize}

También hay que tenr en cuenta que: - A medida que el disolvente se
evapora, la mezcla se vuelve ``más pesada'' y la temperatura de
ebulición sube (esto es crítico para no degradar el biodiesel). - La
energía necesaria para evaporar (\(\Delta H_{vap}\)) cambia según la
composición.

En caso de necesitar incorporar el uso de la destilación en la
simulación del reciclado de los ésteres y las microemulsiones, se
recomienda utilizar el código \texttt{destilacion\_fames.py} como base
para la simulación de la columna de destilación, adaptándolo según las
necesidades específicas del proceso y las características de la mezcla a
tratar.

La simulación \texttt{destilacion\_fames.py} modela una columna de
destilación para purificar el biodiésel recuperado. Utiliza un enfoque
de equilibrio de fases y balances de masa para determinar la composición
del destilado y del fondo en función de las condiciones operativas de la
columna. Ambos códigos están diseñados para ser flexibles y adaptarse a
diferentes condiciones de operación y composiciones de alimentación,
permitiendo a los usuarios optimizar el proceso de reciclaje de
biodiésel.

En el mundo real, a medida que el disolvente (metanol/hexano) se
evapora, el líquido restante se vuelve ``más espeso'' (más viscoso). En
un ATFE, las palas del rotor giran para mantener el líquido pegado a la
pared en una fina capa: - Si la viscosidad es baja (agua/metanol): el
flujo es muy turbulento, la transferencia de calor (\(U\)) es muy alta.
- Si la viscosidad es alta (fames/resinas): el líquido se congela, la
turbulencia es baja y cuesta mucho más calentar el interior. El valor de
\(U\) se desploma.

\subsection{\texorpdfstring{Física: cómo modelar la variable
\(U\)}{Física: cómo modelar la variable U}}\label{fuxedsica-cuxf3mo-modelar-la-variable-u}

En ingeniería química, para equipos de superficie rayada (como ATFE), se
suele utilizar la teoría de la penetración o las correlaciones
empíricas. Un enfoque estándar simplificado es escalar la base \(U\) en
función de la viscosidad:

\[
U_{local} = U_{base} \cdot \left( \frac{\mu_{base}}{\mu_{local}} \right)^n
\]

Donde \(n\) suele estar entre \(0,25\) y \(0,33\). Esto significa que si
la viscosidad sube mucho, el \(U\) baja, pero no de forma lineal (las
comillas ayudan a mitigar la caída). Código Python actualizado (con
viscosidad dinámica) Agregamos la lógica para leer la viscosidad
(mix.mul) en cada paso y recalcular la eficiencia térmica y creamos una
nueva simulación \texttt{afte\_2.py} que tiene en cuenta la varción de
viscosidad.

Cuando se ejecute este código se revela en el gráfico: - La ``curva de
agotamiento'' de \(U\): al principio, \(U\) es alto
(\textasciitilde1500). Pero hacia el final, cuando el biodiesel es casi
puro (90\%+), la línea roja (viscosidad) se dispara y la línea azul
(\(U\)) cae. - Consecuencia práctica: Los últimos 0,5 metros del
evaporador son muy ineficientes. Es mucho más difícil sacar el último
1\% del disolvente que el primer 50\%.

Lo siguiente fue preparar una simulación a nivel ``industrial'' contando
con el Tiempo de Residencia. El biodiesel se degrada si pasa mucho
tiempo caliente. Pero ante tendremos en cuenta algunas consideraciones
teóricas.

\chapter{Simulaciones de recuperación de FAMES y microemulsiones en
ASPEN - HYSYS y
DWSIM}\label{simulaciones-de-recuperaciuxf3n-de-fames-y-microemulsiones-en-aspen---hysys-y-dwsim}

\section{El Biodiésel como Biosolvente
Industrial}\label{el-biodiuxe9sel-como-biosolvente-industrial}

El biodiésel (FAME - Ésteres Metílicos de Ácidos Grasos) se conoce
principalmente como combustible, pero en la industria química moderna
está ganando terreno como ``Biodisolvente Verde''. Se utiliza para
sustituir disolventes derivados del petróleo (como el tolueno o xileno)
en la limpieza de resinas, tintas y formulación de agroquímicos debido
a: - Alto punto de inflamación: Es más seguro de manipular. -
Biodegradabilidad: Menor impacto en caso de vertido. - Poder disolvente
(Índice Kauri-Butanol): Muy eficaz para disolver compuestos orgánicos no
polares.

El reto económico y ambiental es que, tras su uso, el biodiésel queda
contaminado. Tirarlo es caro y contaminante; reciclarlo es obligatorio.

\section{Tecnología Clave: Evaporador de Película Agitada
(ATFE)}\label{tecnologuxeda-clave-evaporador-de-peluxedcula-agitada-atfe}

Para reciclar el biodiésel, debemos separarlo de los contaminantes
(pinturas, polímeros, aceites sucios). La destilación convencional (en
tanque o columna) es peligrosa para el biodiésel porque requiere
mantenerlo caliente mucho tiempo, lo que provoca su degradación. Aquí
entra el ATFE (Agitated Thin Film Evaporator). ¿Qué es y cómo funciona?:

El ATFE es un intercambiador de calor de carcasa y tubos modificado.
Dentro del tubo hay un rotor con aspas que gira a alta velocidad. - El
líquido entra y las aspas lo ``esparcen'' contra la pared caliente
formando una película finísima (0.5 - 2 mm). - Al ser una capa tan fina,
la transmisión de calor es instantánea. Los componentes volátiles se
evaporan en segundos. - El residuo pesado (o el producto purificado,
dependiendo del proceso) cae por gravedad en espiral hacia el fondo.

\subsection{Ventajas Críticas para el
Biodiésel:}\label{ventajas-cruxedticas-para-el-biodiuxe9sel}

\begin{itemize}
\tightlist
\item
  Tiempo de Residencia Corto: El líquido solo está dentro del equipo
  unos segundos (10-60 s). Esto evita que el biodiésel se ``queme'' u
  oxide.
\item
  Manejo de Viscosidad: Como viste en la simulación, cuando el
  disolvente se evapora, la viscosidad sube. Las aspas del ATFE rompen
  esa viscosidad, manteniendo el flujo.
\item
  Operación a Alto Vacío: Permite bajar drásticamente la temperatura de
  ebullición.
\end{itemize}

\subsection{Tipos de ATFE:}\label{tipos-de-atfe}

\begin{itemize}
\tightlist
\item
  Vertical (Más común): Usa la gravedad para bajar el producto. Ideal
  para destilar el solvente limpio.
\item
  Horizontal: Se usa cuando se quiere secar completamente un sólido
  (formar polvo o pasta muy seca).
\item
  Rotores Rígidos vs.~Pendulares:

  \begin{itemize}
  \tightlist
  \item
    Rígidos: Mantienen una distancia fija con la pared (gap fijo).
  \item
    Pendulares: La fuerza centrífuga pega las aspas a la pared (scrape
    surface), ideal si el residuo es muy pegajoso (fouling).
  \end{itemize}
\end{itemize}

\section{Comparativa: Girasol
vs.~Palma}\label{comparativa-girasol-vs.-palma}

No todos los biodiéseles son iguales. Su comportamiento en el reciclado
y su impacto ambiental dependen de su Perfil de Ácidos Grasos.

Diferencias Químicas y Operativas en el ATFE

\begin{longtable}[]{@{}
  >{\raggedright\arraybackslash}p{(\linewidth - 4\tabcolsep) * \real{0.3607}}
  >{\raggedright\arraybackslash}p{(\linewidth - 4\tabcolsep) * \real{0.3443}}
  >{\raggedright\arraybackslash}p{(\linewidth - 4\tabcolsep) * \real{0.2951}}@{}}
\toprule\noalign{}
\begin{minipage}[b]{\linewidth}\raggedright
Característica
\end{minipage} & \begin{minipage}[b]{\linewidth}\raggedright
Biodiésel de Girasol
\end{minipage} & \begin{minipage}[b]{\linewidth}\raggedright
Biodiésel de Palma
\end{minipage} \\
\midrule\noalign{}
\endhead
\bottomrule\noalign{}
\endlastfoot
Composición Principal & Rico en Linoleico (C18:2). Es poliinsaturado
(muchos dobles enlaces) & Rico en Palmítico (C16:0) y Oleico. Es
saturado (cadenas estables) \\
Estado Físico (20°C) & Líquido fluido & Semisólido o pasta (se congela
fácilmente) \\
Punto de Niebla (CFPP) & Bajo (-5°C a 0°C). Fluye bien en frío & Alto
(+13°C a +15°C). Necesita calefacción \\
Estabilidad Térmica & BAJA. Los dobles enlaces reaccionan con el oxígeno
y el calor. Tiende a polimerizar (formar gomas) dentro del ATFE si se
pasa de temperatura & ALTA. Resiste muy bien el calor sin degradarse. Es
más robusto para reciclar \\
Reto en el Reciclado & Evitar la oxidación. Requiere vacío más profundo
y temperaturas más bajas en el ATFE & Evitar que se solidifique.
Requiere eléctrico (calefacción) en todas las tuberías y bombas \\
\end{longtable}

\section{Diagrama de Bloques del Proceso de
Reciclado}\label{diagrama-de-bloques-del-proceso-de-reciclado}

\pandocbounded{\includegraphics[keepaspectratio]{../03_Recursos_Graficos/diagrama_bloques_reciclado_biodiesel.png}}

\section{Implicaciones Ambientales (LCA - Análisis de Ciclo de
Vida)}\label{implicaciones-ambientales-lca---anuxe1lisis-de-ciclo-de-vida}

El origen de la materia prima define la ``mochila ecológica'' del
disolvente reciclado.

\section{Biodiésel de Girasol}\label{biodiuxe9sel-de-girasol}

\begin{itemize}
\tightlist
\item
  Impacto de Uso de Suelo: Se cultiva extensivamente en Europa (España,
  Francia, Ucrania). Su huella de transporte es baja si la planta de
  reciclado está en Europa.
\item
  Huella de Carbono: Generalmente menor debido al transporte reducido.
\item
  Crítica: Menor rendimiento por hectárea que la palma, requiere más
  tierra para producir la misma cantidad de aceite.
\end{itemize}

\section{Biodiésel de Palma}\label{biodiuxe9sel-de-palma}

Es un aceite que podríamos denominar eficiente pero polémico.-
Rendimiento: Es el cultivo oleaginoso más eficiente del mundo (produce
5-10 veces más aceite por hectárea que el girasol). Impacto Ambiental:
Históricamente asociado a la deforestación de selvas tropicales
(Indonesia, Malasia) y pérdida de biodiversidad (orangutanes).
Certificación: Si se usa palma, industrialmente se exige certificado
RSPO (Roundtable on Sustainable Palm Oil) para asegurar que no proviene
de deforestación reciente. Transporte: Alta huella de carbono logística
para traerlo a plantas de reciclaje en Occidente.

\section{Resumen del Proceso de
Reciclado}\label{resumen-del-proceso-de-reciclado}

El diagrama de bloques del proceso completo sería: - Recepción:
Biodiésel sucio (con tintas/resinas) - Filtración Previa: Eliminar
virutas metálicas o sólidos grandes (\textgreater100 micras) para no
dañar el rotor del ATFE. - Evaporación Flash (Opcional): Si trae agua o
disolventes muy ligeros (acetona), se quitan aquí. - Unidad ATFE
(Corazón del proceso): - Entrada: Biodiésel Sucio.Condiciones: Vacío
(10-50 mbar) y Tª (160-180°C). - Salida Vapor: Biodiésel puro (se
condensa y recupera). - Salida Fondos: Lodos concentrados (residuos de
resina/pintura) para gestión de residuos externos. - Control de Calidad:
Comprobar acidez y contenido de agua para validar su reutilización.

\subsection{¿Cómo te contempla esto en una simulación
Python?}\label{cuxf3mo-te-contempla-esto-en-una-simulaciuxf3n-python}

Hemos refinado el código Python: - Si simulamos Girasol: Debemos vigilar
que la temperatura de pared (\(T_{wall}\)) no sea excesiva, o añadir una
función de ``pérdida por degradación''. - Si simulamoss Palma: No hay
que preocuparse tanto por la degradación, pero se debe vigilar la
Viscosidad a temperaturas bajas (si el líquido se enfría, la viscosidad
se dispara exponencialmente y bloquea el equipo).

\chapter{Reciclado de ésteres
dibásicos}\label{reciclado-de-uxe9steres-dibuxe1sicos}

\section{¿Qué son los Ésteres Dibásicos
(DBE)?}\label{quuxe9-son-los-uxe9steres-dibuxe1sicos-dbe}

Los DBE (Dibasic Esters) son una familia de disolventes oxigenados
derivados de ácidos dicarboxílicos. Industrialmente, rara vez se usan
puros; suelen ser una mezcla de tres ésteres metílicos purificados (la
fracción ``AGS''): - Dimetil Adipato (C6) - Dimetil Glutarato (C5) -
Dimetil Succinato (C4)

¿Por qué son ``Disolventes Verdes''?: - Baja Presión de Vapor: Se
evaporan muy lentamente (bajos COV - Compuestos Orgánicos Volátiles), lo
que reduce la inhalación por parte de los trabajadores. -
Biodegradabilidad: Se degradan fácilmente en el medio ambiente. - No son
inflamables (generalmente): Tienen puntos de inflamación altos
(\(>100^\circ\text{C}\)), similares al biodiésel. - Sustitución: Se usan
para reemplazar disolventes tóxicos o clorados (como el Cloruro de
Metileno o la NMP) en decapado de pinturas, limpieza de resinas y
limpieza de moldes de poliuretano.

\section{El Reciclado de DBEs}\label{el-reciclado-de-dbes}

Recuperar DBE es muy rentable porque es un disolvente caro (más que el
biodiésel). \#\#\#\# Proceso en ATFE

El comportamiento en un evaporador de película agitada es excelente,
pero con una diferencia clave respecto al biodiésel: - Punto de
Ebullición:Los DBE hierven entre 196°C y 225°C (a presión atmosférica).
Esto es más bajo que el biodiésel (\(>300^\circ\text{C}\)), pero mucho
más alto que el agua o el acetona. - Hidrólisis (El enemigo oculto): Si
el residuo sucio contiene agua y el pH es ácido o básico, al calentar en
el ATFE, el éster se rompe (hidrólisis), volviendo a formarse ácido
adípico/glutárico (que son corrosivos) y metanol. - Solución: Es
obligatorio neutralizar el pH y eliminar el agua (Flash previo) antes de
meterlo al ATFE caliente.

\section{La Mezcla DBE + FAME: Retos de
Interacción}\label{la-mezcla-dbe-fame-retos-de-interacciuxf3n}

En la industria es común encontrar mezclas (blends) de DBE + FAME. Se
mezclan a propósito para obtener un disolvente con ``doble acción'': el
DBE ataca la resina y el Biodiésel aporta lubricidad y mantiene la
suciedad en suspensión.

Al intentar reciclar esta mezcla, surgen complicaciones termodinámicas:

\subsection{Diferencia de Volatilidad
(Separación)}\label{diferencia-de-volatilidad-separaciuxf3n}

Termodinámicamente, no son difíciles de separar entre sí si tienes una
buena columna, pero en un ATFE (que tiene solo una etapa teórica de
equilibrio) es complejo obtener cortes puros.DBE: - Hierve a
\(\approx 200^\circ\text{C}\). - FAME: Hierve a
\(\approx 350^\circ\text{C}\).

En un solo paso de ATFE: - Si se ajusta la temperatura para evaporar el
DBE, el biodiésel se quedará en el fondo con los residuos (se pierde
biodiésel). - Si se sube la temperatura para evaporar también el
biodiésel, ambos saldrán mezclados por el tope (recuperas el disolvente
mezcla, pero no los separas). - Nota: A menudo, recuperar la mezcla
``Sucia pero destilada'' es aceptable para volver a usarla en limpieza
industrial.

\subsection{Azeótropos y
No-Idealidad}\label{azeuxf3tropos-y-no-idealidad}

En este caso es donde la simulación (UNIQUAC/NRTL) es vital. -
Azeótropos con Agua: Tanto los DBE como el Biodiésel forman azeótropos
con el agua. Esto significa que es difícil secarlos completamente solo
con evaporación; siempre arrastran humedad, lo que fomenta la corrosión
ácida. - Interacción DBE-Biodiésel: - Químicamente son muy similares
(polares, ésteres). Generalmente forman una mezcla zeotrópica (no
azeotrópica), lo cual es bueno. Siguen la Ley de Raoult con desviaciones
moderadas. - El riesgo de la Transesterificación Cruzada: - Si la mezcla
caliente contiene trazas de catalizador (ej. sosa cáustica de una
limpieza anterior o metóxido), los grupos metilo pueden ``saltar'' de
una molécula a otra. Aunque termodinámicamente no cambia mucho la mezcla
(ambos son metil-ésteres), químicamente estás creando moléculas híbridas
impredecibles. \#\#\#\# Polimerización Cruzada

Si el residuo contiene pinturas o resinas de poliuretano: - Los DBE son
excelentes disolventes de poliuretano. - Al concentrar el residuo en el
fondo del ATFE, la concentración de resina sube. - A altas temperaturas,
el biodiésel (especialmente el de girasol) puede actuar como agente de
curado o polimerizar con los residuos disueltos por el DBE, creando un
gel sólido que puede bloquear el rotor del ATFE.

\section{Implementación en la Simulación
Python}\label{implementaciuxf3n-en-la-simulaciuxf3n-python}

La incorporación de los ésteres dibásicos (DBE) a la simulación es un
paso importante porque pasamos de una mezcla binaria (Disolvente/FAME) a
una Mezcla Ternaria (o Multicomponente). El comportamiento cambia
drásticamente: - El Ligero (ej. Metanol/Agua): Se evapora casi
instantáneamente al entrar. - El Medio (DBE - Dimetil Adipato): Es el
``componente difícil''. Tiene un punto de ebullición alto
(\textasciitilde227°C). Si nos pasamos de calor, lo evaporamos (se
pierde por arriba); si nos quedam os cortos, se queda en el biodiésel. -
El Pesado (Biodiésel - Oleato): Debería quedarse líquido.

Para simular esto, se debe añadir un componente representativo de los
DBE. El Dimetil Adipato es el mejor candidato estándar. Podríiamos crear
un nuevo archivo \texttt{atfe\_dbe\_fame.py} que incluya: - Componentes:
FAME (oleato de metilo) + Dimetil Adipato + Agua + Contaminante (tolueno
o resina). - Modelo Termodinámico: NRTL o UNIQUAC para capturar las
interacciones. - Lógica de Hidrólisis: Si el pH \textless{} 6 o
\textgreater{} 8 y hay agua, añadir una ``pérdida'' de DBE proporcional
a la temperatura y tiempo de residencia. - Control de Viscosidad:
Similar al código anterior, pero considerando la mezcla DBE-FAME.\\
- Salida: Composición del vapor (recuperado) y del fondo (residuo).

\subsection{\texorpdfstring{El Reto Matemático: Composición del Vapor
(\(y_i\))}{El Reto Matemático: Composición del Vapor (y\_i)}}\label{el-reto-matemuxe1tico-composiciuxf3n-del-vapor-y_i}

En el código anterior asumíamos que solo evaporaba el disolvente. Ahora,
todo puede evaporarse según su presión de vapor.Usaremos el cálculo de
equilibrio de fases (\(y_i\) vs \(x_i\)) de \texttt{thermo} para
determinar qué composición exacta tiene el vapor que estamos retirando
en cada centímetro del tubo.

Hemos creado otro simulador \texttt{atfe\_dbe\_fame\_vle.py} que incluye
este cálculo de equilibrio. En cada paso: - Calculamos la composición
líquida actual (\(x_i\)). - Usamos \texttt{thermo} para obtener las
presiones de vapor parciales y calcular la composición del vapor
(\(y_i\)). - Ajustamos las tasas de evaporación de cada componente según
su fracción en el vapor.

Con estas consideraciones, el código permitirá simular y optimizar el
proceso de reciclado de mezclas DBE-FAME en un ATFE, ayudando a
maximizar la recuperación de disolventes y minimizar los riesgos
operativos.

\subsection{Interpretación de los
Resultados}\label{interpretaciuxf3n-de-los-resultados}

Cuando se ejecuta este código, en la Gráfica 1 (Composición) se verás la
``guerra'' de volatilidades relativas: - Metanol (Línea Roja punteada):
- Caerá a cero muy rápido (en los primeros 0.5 o 1 metro). - Es tan
volátil que el vacío se lo lleva enseguida. - DBE (Línea Azul): - Al
principio, su concentración sube. ¿Por qué? Porque estamos quitando
metanol, así que porcentualmente el DBE y el Biodiésel aumentan. - Una
vez que el metanol desaparece, la temperatura sube. Si la \(T_{pared}\)
es suficientemente alta, empezarás a ver cómo la línea azul empieza a
bajar. Eso significa que estás evaporando el DBE (recuperándolo por
arriba). - Biodiésel (Línea Verde): Debería subir constantemente hasta
acercarse al 100\% (o al valor que desees) en el fondo.

\subsection{Experimentación: ¿Cómo ``jugar'' con el
código?}\label{experimentaciuxf3n-cuxf3mo-jugar-con-el-cuxf3digo}

Si se quiere separar el DBE del Biodiésel: - Subir T\_wall\_C a 190°C o
200°C. y la línea azul (DBE) caerá al final del tubo (se va al vapor). -
Si quieres limpiar el disolvente (DBE+Biodiésel) de impurezas
ligeras:Baja T\_wall\_C a 140°C. El Metanol desaparecerá, pero el DBE se
mantendrá líquido junto con el Biodiésel (la línea azul se mantendrá
alta).

\subsubsection{Nota sobre Azeótropos}\label{nota-sobre-azeuxf3tropos}

El código actual usa thermo, que tiene modelos de coeficientes de
actividad (como UNIFAC o NRTL modificados internamente si están
disponibles). Si existen azeótropos conocidos en la base de datos entre
estos componentes, el cálculo mix.flash ya los está teniendo en cuenta
implícitamente.Ejemplo: Si hubiera un azeótropo Metanol-DBE, verías que
no puedes bajar la concentración de uno de ellos a cero sin arrastrar el
otro.

La simulación se mejoró también par acalcular la potencia consumida en
el proceso. El cálculo de la potencia del rotor es fundamental. En
plantas reales de reciclado de biodiésel, este es a menudo el ``cuello
de botella''.

Si el residuo se vuelve demasiado viscoso (como una goma o pegamento) al
final del tubo, el motor eléctrico sufrirá. Si no está bien
dimensionado, saltarán las protecciones térmicas y la planta se parará.

\section{La Física: ¿Cuánta energía cuesta mover las
aspas?}\label{la-fuxedsica-cuuxe1nta-energuxeda-cuesta-mover-las-aspas}

El consumo de potencia en un ATFE se debe principalmente al
Cizallamiento Viscoso (arrastrar las palas a través del líquido).La
fórmula de ingeniería simplificada para cada sección del tubo es:

\[
P_{corte} = \mu \cdot A \cdot \frac{v^2}{\delta}
\]

Donde:\(\mu\): Viscosidad dinámica local (Pa·s) {[}Calculada por
thermo{]}.\(A\): Área de la sección (\(m^2\)).\(v\): Velocidad
periférica de la pala (\(m/s\)).\(\delta\): Espesor de la película o
``gap'' entre pala y pared (m).

El código \texttt{atfe\_dbe\_fame\_poterncia.py} incluye el cálculo del
consumo energético del ATFE basado en la transferencia de calor
necesaria para evaporar los componentes volátiles. Este cálculo es
crucial para evaluar la viabilidad económica del proceso de reciclado.

\section{Interpretación de la
simulación}\label{interpretaciuxf3n-de-la-simulaciuxf3n}

\subsection{La correlación
Viscosidad-Potencia}\label{la-correlaciuxf3n-viscosidad-potencia}

La línea roja (Potencia) debiera ser casi idéntica en forma a la azul
(Viscosidad). Esto confirma que el consumo eléctrico es un ``sensor''
indirecto de cuán espeso está el producto.

\subsection{El peligro del fondo}\label{el-peligro-del-fondo}

Si la línea roja se dispara exponencialmente en los últimos 0.5 metros,
tienes un problema de diseño,lo que significa que el residuo se está
volviendo demasiado viscoso., o sea que se está secando demasiado el
residuo, lo cual es un riesgo ya que los FAMEs polimerizados puede
bloquear el rotor (stall) o quemar el motor. La solución podría ser
dejar un poco más de DBE o Biodiésel en el fondo (no intentar llegar al
100\% de pureza) para que actúe como lubricante.

\subsection{Tabla Comparativa
Técnica}\label{tabla-comparativa-tuxe9cnica}

Los resultados que obtendrías en la simulación (usando methyl linoleate
vs methyl palmitate) con la realidad operativa en planta permiten
elaborar la siguiente tabla comparativa.

\begin{longtable}[]{@{}
  >{\raggedright\arraybackslash}p{(\linewidth - 4\tabcolsep) * \real{0.1722}}
  >{\raggedright\arraybackslash}p{(\linewidth - 4\tabcolsep) * \real{0.4067}}
  >{\raggedright\arraybackslash}p{(\linewidth - 4\tabcolsep) * \real{0.4211}}@{}}
\toprule\noalign{}
\begin{minipage}[b]{\linewidth}\raggedright
Variable de Ingeniería
\end{minipage} & \begin{minipage}[b]{\linewidth}\raggedright
Biodiésel de Girasol
\end{minipage} & \begin{minipage}[b]{\linewidth}\raggedright
Biodiésel de Palma
\end{minipage} \\
\midrule\noalign{}
\endhead
\bottomrule\noalign{}
\endlastfoot
Componente Clave (Simulación) & Metil Linoleato (C18:2) & Metil
Palmitato (C16:0) \\
Estructura Molecular & Cadena larga con 2 dobles enlaces (``kinks'').
Molécula ``doblada''. & Cadena saturada lineal. Molécula ``recta'' y
empaquetada. \\
Viscosidad (100∘C) & Baja. Fluye muy bien incluso concentrado. &
Media/Alta. Al ser saturado, es más espeso. \\
Punto de Fusión / Fluidez & −5∘C (Líquido siempre). & +30∘C (Tiende a
solidificar en líneas frías). \\
Sensibilidad Térmica (Degradación) & Muy Alta. Los dobles enlaces
reaccionan con el calor. & Baja. Muy estable y robusto. \\
Riesgo Principal en ATFE & Fouling por Polimerización. Si
Tpared\hspace{0pt}\textgreater180∘C, se forman gomas (barniz) en la
pared. & Bloqueo por Enfriamiento. Si el residuo se enfría al salir, se
hace una ``vela'' sólida. \\
Potencia del Rotor (Simulada) & Menor consumo (el líquido lubrica bien).
& Mayor consumo (+15-20\%) debido a la mayor viscosidad. \\
Estrategia de Operación & Maximizar Vacío (\textless10 mbar) para bajar
la Tª de ebullición. & Mantener traceado eléctrico en la descarga de
fondo. \\
\end{longtable}

\section{Simulación con aceite de palma en vez de
girasol}\label{simulaciuxf3n-con-aceite-de-palma-en-vez-de-girasol}

La palma es rica en ácidos saturados. Usaremos el Metil Palmitato.

Simplemente habría que cambiar el el código python lo siguiente:

\begin{verbatim}
# Escenario PALMA
# Sustituimos 'methyl oleate' por 'methyl palmitate'
comps = ['methanol', 'dimethyl adipate', 'methyl palmitate']

# Nota para la simulación:
# Verás que la viscosidad (mu_liq) calculada por 'thermo' será mayor.
# Esto hará que el cálculo de 'Power_segment_Watts' suba.
\end{verbatim}

\subsection{Interpretación de
Resultados}\label{interpretaciuxf3n-de-resultados}

Si se ejecutan ambos escenarios y se comparan las gráficas de Potencia,
se puede concluir lo siguiente: - Eficiencia Energética: El reciclado de
biodiésel de girasol consume menos electricidad en el motor del ATFE
porque el fluido ofrece menos resistencia al cizallamiento. - Ventana
Operativa: - El Girasol tiene una ventana estrecha por arriba (límite de
temperatura). - La Palma tiene una ventana estrecha por abajo (límite de
solidificación). - Calidad del Destilado: Debido a la estabilidad, es
más probable obtener un disolvente reciclado de Palma con menos acidez
(menos degradación) que uno de Girasol, a menos que el equipo de vacío
funcione perfectamente.

\chapter{Conclusiones}\label{conclusiones}

\section{Conclusiones del Estudio}\label{conclusiones-del-estudio}

El presente informe evalúa la viabilidad técnica de la recuperación
biosolventes (Biodiésel y Ésteres Dibásicos) mediante evaporación en
película agitada (ATFE). A través del modelado computacional en Python,
integrando termodinámica de equilibrio de fases y mecánica de fluidos,
se han alcanzado las siguientes conclusiones: - Idoneidad de la
Tecnología ATFE:La simulación confirma que el ATFE es el equipo crítico
e insustituible para este proceso. La combinación de alto vacío
(\textless50 mbar) y tiempos de residencia cortos es la única vía para
separar los contaminantes sin provocar la degradación térmica
(oxidación) del biodiésel ni la hidrólisis de los ésteres dibásicos
(DBE). - Influencia Crítica de la Materia Prima (Girasol vs.~Palma):El
perfil de ácidos grasos determina la estrategia operativa: - El
Biodiésel de Girasol (Linoleico) presenta ventajas mecánicas (menor
viscosidad y consumo de potencia en el rotor), pero impone un límite
térmico estricto (\(T_{pared} < 170^\circ\text{C}\)) debido a su alta
tendencia a la polimerización en caliente. - El Biodiésel de Palma
(Palmítico) ofrece una gran estabilidad térmica, permitiendo operaciones
más agresivas, pero requiere un mayor consumo energético en el motor
(+15-20\%) y sistemas de calentamiento auxiliar (traceado) para evitar
la solidificación del residuo en la descarga. - Dinámica de la
Recuperación de DBEs:La recuperación de Ésteres Dibásicos en mezcla con
biodiésel es termodinámicamente viable pero compleja. La simulación
demuestra que existe una ``ventana operativa'' estrecha: se requiere una
temperatura de pared suficiente para evaporar el DBE (P.eb.
\(\approx 200^\circ\text{C}\)), pero el aumento de viscosidad resultante
al secar el fondo reduce drásticamente el coeficiente de transferencia
de calor (\(U\)), limitando la eficiencia en los últimos tramos del
evaporador. - Consumo Energético:El modelo electromecánico desarrollado
revela que el consumo de potencia del rotor no es lineal. Se dispara
exponencialmente en el último tercio del equipo a medida que aumenta la
concentración de sólidos y la viscosidad. Esto subraya la importancia de
no intentar secar el residuo al 100\%, dejando un remanente de
disolvente como lubricante para proteger el equipo.5.2.

\section{Recomendaciones para Trabajos
Futuros}\label{recomendaciones-para-trabajos-futuros}

Para avanzar desde esta simulación teórica hacia una implementación
industrial, se sugieren las siguientes líneas de acción: - Validación
Experimental en Planta Piloto: Se recomienda realizar ensayos con
muestras reales de disolvente sucio en un ATFE a escala laboratorio
(\(0.05 - 0.1 m^2\)) para validar los coeficientes de transferencia de
calor (\(U\)) calculados y ajustar el exponente de viscosidad (\(n\))
del modelo. - Análisis de Ciclo de Vida (LCA): Realizar un estudio
comparativo de la huella de carbono entre el proceso de reciclado
propuesto y la incineración del residuo. El estudio debe cuantificar si
el ahorro de emisiones por no producir disolvente virgen compensa el
consumo eléctrico de las bombas de vacío y el calentamiento del ATFE. -
Diseño del Sistema de Vacío: Dado que se evaporan compuestos orgánicos
volátiles (VOCs), se recomienda estudiar la implementación de bombas de
vacío de tornillo seco en lugar de anillo líquido, para evitar la
generación de agua residual contaminada. - Integración Energética:
Evaluar la posibilidad de utilizar el calor latente de los vapores de
disolvente recuperados (que salen a \(>100^\circ\text{C}\)) para
precalentar la alimentación, mejorando la eficiencia global de la
planta.




\end{document}
